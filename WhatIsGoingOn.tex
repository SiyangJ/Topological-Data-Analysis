\documentclass[10pt,a4paper]{article}
\usepackage[utf8]{inputenc}
\usepackage{amsmath}
\usepackage{amsfonts}
\usepackage{amssymb}
\begin{document}
\part{Image}
\section{Segmentation}
\subsection{Qaiser 2017}
This paper \cite{Qaiser2017} applies 1) persistent homology to extract features, 2) CNN to classify, and 3) Ensemble Random Forest to tumor segmentation problems.\\
\subsubsection{Motivation}
\paragraph{Different Topological Structure} for tumor and normal regions. Specifically, degree of connectivity is different. So quantify these features by PH.
\subsubsection{Method}
\paragraph{Overview}
Use CNN to divide to patches, PH to create PH profiles, use KLD (Kullback-Leibler divergence) as a distance function between the profiles to perform kNN.
\section{Classification}
\subsection{Chittajallu 2018}
This paper \cite{Chittajallu2018} uses persistence landscape and persistence image to train machine learning models.
\part{3D Object}

\subsection{Bendich 2016}
\newpage
\bibliography{Reference}
\bibliographystyle{ieeetr}

\end{document}