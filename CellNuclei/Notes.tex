\documentclass[10pt,a4paper]{article}
\usepackage[utf8]{inputenc}
\usepackage{amsmath}
\usepackage{amsfonts}
\usepackage{amssymb}


\begin{document}

\section{General Notes}
\subsection{Image}
\subsubsection{Image Types}
\begin{enumerate}
	\item Pathology images
	\item Fluorescence microscopy
	\item Confocal images
	\item H\&E stained cancer images
\end{enumerate}
\subsubsection{Microscopy Models}
\begin{enumerate}
	\item Bright field
	\item Fluorescence
	\item Phase contrast
\end{enumerate}

\subsection{Random Summary}
Seems a lot of the problems w/ different models are parameter selection. And usually, we can use some adaptive methods to optimally choose the parameter.\\
Often, can be converted to optimization problems. And often can be exactly solved using DP like methods.




\section{\cite{Chen2013}}

\section{\cite{Xing2016}}

\subsection{Detection}

\subsubsection{Distance Transform}
\paragraph{Mechanism}Local maxima = centroids of nuclei or cells. Often paired with "Watershed Segmentation".
\paragraph{Advantage}
\paragraph{Disadvantage}Only effective on regular shapes in a binary image. Susceptible to small changes. Complex image $\to$ variations $\to$ over-detection.
\paragraph{Improvement}Gaussian filter, then trace gradient vector field. Accumulated pixels threshold to distinguish b/w local and non-local maxima.
\paragraph{Further}Lin et al. gradient weighted-distance transform for 3D fluorescence image.

\subsubsection{Morphology Operation}
\paragraph{Mechanism}Binary morphological filtering for images w/ certain structure element, circle, square, cross... Examining the geometrical and topological structures of objects w/ predefined shape. Four basic shift-invariant operators:
\begin{enumerate}
	\item Erosion
	\item Dilation
	\item Opening
	\item Closing
\end{enumerate}
The four can be used to generate more basic morphological operations, boundary, hole, skeletonizing... Binary morphology can be extended to gray-scale morphology. Widely used operators: top-hat, bottom-hat transforms. For example, UE (Ultimate Erosion). Erosion until can't.
\paragraph{Advantage}Can be used to basic image enhancement, preparing for further analysis. UE: can separate touching or overlapping cells.
\paragraph{Disadvantage}UE: can produce multiple marker for each cell.
\paragraph{Improvement}
\begin{enumerate}
	\item Improved UE, Park et al. noise robust stopping criterion. Perform until convex. However, binarization.
	\item Conditional Erosion: Yang et al. Coarse erosion preserves shapes, and fine erosion avoids under-segmentation
\end{enumerate}
\paragraph{Further}
\subparagraph{}Hodneland 3D fluorescence images. Adaptive threshold for ridge extraction, then link gaps.
\subparagraph{}Plissiti gray-scale, not converting.

\subsubsection{H-minima/maxima Transform}
\paragraph{Mechanism}
Based on morphology operation, used in local minima detection. Image $A$, depth value $h$, $H(A,h)=R^{\epsilon}(A+h)$, where $R^{\epsilon}$ is reconstruction by erosion. Some regional minima are suppressed. Initially connected parts can be split in terms of the detected minima, $h$ leads to under/over-segmentation. Usually used to generate markers for watershed transform based segmentation.
\paragraph{Advantage}
Compared with DT (EDT), all minima $\to$ H-minima. Very popular in biomedical images.
\paragraph{Disadvantage}
Suppress minima, so needs enhancement beforehand. Properly defined $h$ value is needed.
\paragraph{Improvement}
\begin{enumerate}
	\item Adaptive HIT., iteratively increase $h$ until a region merging. Ignores nucleus size.
	\item Jung and Kim, adaptively choose $h$ to minimize segmentation distortion.
	\item Variance in cell areas.
\end{enumerate}
\paragraph{Further}

\subsubsection{LoG, Laplacian of Gaussian}
\paragraph{Mechanism}
In medical image analysis, LoG is one of the most popular for small blobs.
\paragraph{Advantage}
\paragraph{Disadvantage}
\begin{enumerate}
	\item Might fail in touching / overlapping objects. 
	\item Scale issue.
\end{enumerate}
\paragraph{Improvement}
\begin{enumerate}
	\item Lindeberg introduces normalizing factor for multiscale LoG blob detector.
	\item Kong generalized LoG, for elliptical structures (oblique elliptical Gaussian)
	\item Hessian analysis to identify optimal scale
	\item Unsupervised GMM can be used to refine blobs
	
\end{enumerate}
\paragraph{Further}

\subsubsection{Maximally Stable Extremal Regions}
\paragraph{Mechanism}
Maximally Stable Extremal Regions. Set of nested extremal regions based on level sets in the intensity landscape. Local intensity minimum-based criterion.
\begin{enumerate}
	\item Generate sufficient number of extremal regions.
	\item Recognize those regions corresponding to real nuclei or cells.
	\begin{enumerate}
		\item Eccentricity
		\item Blob appearance + shape properties
		\item Arteta formulates an optimization problem, candidates -> scores -> DP for maximal total score
		\item Multilevel thresholding
	\end{enumerate}
\end{enumerate}
\paragraph{Advantage}
\paragraph{Disadvantage}
\paragraph{Improvement}
\paragraph{Further}

\subsubsection{Hough Transformation}
\paragraph{Mechanism}
Circular/elliptical nuclei in pathological images. From $xy$-plane transform to circular $a,b,r$ parameter space. Discrete voting strategy? Most votes corresponding to parameter? Locate the targets by seeking peaks in parameter space (e.g. gradient descent). 
\paragraph{Advantage}
\paragraph{Disadvantage}
False peaks due to noise, incorrect edge extraction, touching objects. Further analysis is needed. 
\paragraph{Improvement}
Gaussian smoothing to denoise, morphology operations to reconstruct. SVM classifier. Optimization problem can be solved by some ILP.
\paragraph{Further}
\begin{enumerate}
	\item Can deal with arbitrary shapes. 
	\item 3D transformation can be done.
	\item Randomized version
\end{enumerate}

\subsubsection{Radial Symmetry Based Voting}
\paragraph{Mechanism}
Locate the centroids of nuclei or cells. High radial symmetry points highlighted.
\paragraph{Advantage}
\paragraph{Disadvantage}
High computational complexity. False peaks due to clustered nuclei. Radius range. What if not circular?
\paragraph{Improvement}
FRST. Candidates, thresholding. Affine transform to deal with non-circular.
\paragraph{Further}

\subsubsection{SVM}

\subsubsection{Random Forest}
\subsubsection{DNN, esp. CNN}
\paragraph{Mechanism}


\subparagraph{Ciregan} Mitotic cell detection in breast cancer histology images.
\begin{enumerate}
	\item Probability map of being centroid of a mitotic cell.
	\item Smoothed w/ disk kernel
	\item Non-maxima suppression
\end{enumerate}
Alternatively, can be formulated into an optimization problem.
\begin{enumerate}
	\item Candidates by LoG, MSER, iterative voting, etc
	\item Score by CNN
	\item Best subset of candidates
\end{enumerate}

\paragraph{Advantage}
\paragraph{Disadvantage}
Computationally expensive for large-scale images.
\paragraph{Improvement}
Fast scanning.
\paragraph{Further}

\subsection{Segmentation}
Methodologies:
\begin{enumerate}
	\item Separate fore and back grounds, and then splits
	\item Markers, then expand
	\item Generate candidates, then select
\end{enumerate}
Algorithms:
\begin{enumerate}
	\item Thresholding
	\item Morphology Operation
	\item Watershed transform
	\item Deformable models
	\item Clustering
	\item Graph-based models
	\item Supervised learning
\end{enumerate}

\subsubsection{Intensity Thresholding}
First and simplest method.
\paragraph{Mechanism}
Assumption: intensity distributions for fore- and back- grounds are sufficiently and consistently distinct. Convert to binary with global threshold, or locally adaptive threshold. Usually empirical. Can also be some optimization problem. Inter-variance for example. 
\paragraph{Advantage}
\paragraph{Disadvantage}
How to choose threshold
\paragraph{Improvement}
Dividing into sub-images. However, introduce other need-to-defined parameters.
\paragraph{Further}
Convert RGB to gray-scale, Callau

\subsubsection{Morphology Operation}
\paragraph{Mechanism}
Top down erosion and bottom up dilation. Erosion until markers are obtained. Grows the markers w/ dilation to reconstruct, while preventing merging.
\paragraph{Advantage}
\paragraph{Disadvantage}
Under-segmentation in dense cell clumps.
\paragraph{Improvement}
Modeling w/ shapes
\paragraph{Further}
Always used to facilitate subsequent segmentation.

\subsubsection{Watershed Transformation}
\paragraph{Mechanism}
\paragraph{Advantage}
\paragraph{Disadvantage}
\paragraph{Improvement}
\paragraph{Further}

\newpage

\bibliography{Reference}
\bibliographystyle{ieeetr}



\end{document}