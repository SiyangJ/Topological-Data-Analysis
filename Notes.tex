\documentclass[10pt,a4paper]{report}
\usepackage[utf8]{inputenc}
\usepackage{amsmath}
\usepackage{amsfonts}
\usepackage{amssymb}
\usepackage{textcomp}

\begin{document}

\section{TODO List}
\subsection{Classic Dimensionality Reduction}
\begin{enumerate}
	\item Learn about the following:
	\begin{enumerate}
		\item PCA
		\item MDS
		\item isomap
	\end{enumerate}
	\item Maybe need to implement some of them
	\item Know the difference from a practical aspect
\end{enumerate}
\subsection{Clustering}
\begin{enumerate}
	\item Need to learn about the different approaches 
	\begin{enumerate}
		\item Variational
		\item Spectral
		\item Hierarchical
		\item Density thresholding
		\item Mode seeking
		\item Valley Seeking
	\end{enumerate}
	\item Need to know the specific applications
	\item Need to delve into one of them to see if possible for improvement
	\item Mathematical details about Mode Seeking
	\begin{enumerate}
		\item Hypotheses on the function of $f$ and estimator $\hat{f}$
		\item Exact conclusions about the prominence gap.
		\item How to specify prominence parameter $\tau$
		\item Algorithm details
	\end{enumerate}
\end{enumerate}

\section{Clustering}
Point cloud (with coordinates)\\

Distance / dissimilarity matrix\\

Note: this seems to be a different idea of the distance function used in TDA, which is called lens in that context.\\

Barcode \textrightarrow merge tree \textrightarrow dendrogram

\subsection{Mode Seeking Paradigm}
Problems:
\begin{enumerate}
	\item Noisy estimator
	\item Neighborhood graph
\end{enumerate}
Solutions:
\begin{enumerate}
	\item Be proactive: smooth
	\item Be reactive: merge clusters after clustering\\
		  This leads to "topological persistence"
\end{enumerate}

Persistence for Model Seeking:\\
Probability density function $f$\\
\begin{itemize}
	\item Nested family (filtration) of of inverse images, or superlevel-sets $f^{-1}([t,+\infty))$ for 			  $t$ from $+\infty$ to $-\infty$
	\item Track evolution of "topology"
\end{itemize}
Similar stability theorem.\\
Seems to have relation with Morse theory. "If $f$ is Morse, then..."\\

\section{Topological Persistence}
Persistence diagram shows the "persistence" of the topological features.
Slight perturbation causes slight difference in persistence diagrams.

\section{Homology}
Definition: $h:X\mapsto Y$ is a homeomorphism if there exists a map $h^{-1}:Y\mapsto X$, s.t.
\begin{itemize}
	\item both are continuous
	\item
\end{itemize}


\section{Mapper}



\end{document}