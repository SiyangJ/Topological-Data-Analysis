\documentclass[10pt,a4paper]{article}
\usepackage[utf8]{inputenc}
\usepackage{amsmath}
\usepackage{amsfonts}
\usepackage{amssymb}
\usepackage{textcomp}
\usepackage{url}

\begin{document}

\section{List of works}
\subsection{Introduction and Summary}
\begin{enumerate}
	\item \cite{Otter2017} Roadmap for computation of PH, 2017\\
		  Thorough review of computation and some theory of PH.

\end{enumerate}
\subsection{Original work}
\subsubsection{TDA}
\begin{enumerate}
	\item \cite{Carlsson2009} Foundation for TDA\\
		  Discusses why topology and functoriality are essential for data analysis.
	\item \cite{Singh2007} Mapper Algorithm
	\item \cite{Carlsson2008} The Natural Image paper.
\end{enumerate}
\subsubsection{Persistent Homology}
\begin{enumerate}
	\item \cite{Mischaikow2013} Discrete Morse theory
	\item \cite{Bubenik2007} Statistical approach to persistent homology
\end{enumerate}
\subsubsection{Computation}
\begin{enumerate}
	\item \cite{Wagner2012} Computation of PH for cubical data.\\
		  Found in \cite{Otter2017}
	\item \cite{Kaczynski2004} Computational homology\\
	talked about cubical complexes and other complexes
	\item \cite{Javaplex} Javaplex
	\item \cite{Fasy2014} Intro to R TDA
	\item \cite{Bubenik2017} Intro of persistence landscapes w/ algorithms.
\end{enumerate}
\subsection{Applications}
\subsubsection{Benchmarking}
\begin{enumerate}
	\item 
\end{enumerate}

\section{Current Challenges}
\subsection{From Data to Filtration}
Which complex is best suited for what?
\subsection{Statistical Interpretation}
Two major challenges are described in \cite{Otter2017}.
\begin{enumerate}
	\item Quantitatively assessing the quality of the barcodes.\\
		  Specifically, one cannot just say I'll disregard the "shorter" ones, and the left are the features.\\
		  How short? What about the variation in length?\\
	\item The space of barcodes lacks geometric properties to define basic concepts\\
		  Mean, median, etc.
\end{enumerate}
\subsubsection{Persistence Diagram}
Few tools can be used in applications for computation of Wasserstein / bottleneck distance between persistence diagrams.\\
Some existing ones include: Dionysus, Hera, TDA Package.
\section{Questions}
\subsection{Theory}
\subsubsection{Homology}
\begin{enumerate}
	\item What other "features" can be revealed by homology?\\
		  It seems that although Homology groups are informational\\
		  Betti numbers, which are the basis for barcode and other diagrams, do not tell us anything more that holes and voids.\\
		  
\end{enumerate}
\subsection{Computation}
\begin{enumerate}
	\item What is the bottleneck of the general computation process?
	\begin{enumerate}
		\item Is it the same for all the libraries?
		\item Possible optimization? Meaningful?
	\end{enumerate}	  
	\item What is the bottleneck for a specific data type?
	\item For a specific purpose?
\end{enumerate}
\subsection{Application}
\begin{enumerate}
	\item Search for a problem?\\
		  It seems that the methods is a not problem-solving-oriented.\\
		  But rather developed from math intuition.\\
		  Therefore seems bound to be less powerful than DL, which is clearly developed to solve problems.\\
		  
\end{enumerate}

\section{TODO List}
\subsection{General Learning Plan}
\begin{enumerate}
	\item Need to read the roadmap\cite{Otter2017} more details.
\end{enumerate}
\subsection{Classic Dimensionality Reduction}
\begin{enumerate}
	\item Learn about the following:
	\begin{enumerate}
		\item PCA
		\item MDS
		\item isomap
	\end{enumerate}
	\item Maybe need to implement some of them
	\item Know the difference from a practical aspect
\end{enumerate}
\subsection{Clustering}
\begin{enumerate}
	\item Need to learn about the different approaches 
	\begin{enumerate}
		\item Variational
		\item Spectral
		\item Hierarchical
		\item Density thresholding
		\item Mode seeking
		\item Valley Seeking
	\end{enumerate}
	\item Need to know the specific applications
	\item Need to delve into one of them to see if possible for improvement
	\item Mathematical details about Mode Seeking
	\begin{enumerate}
		\item Hypotheses on the function of $f$ and estimator $\hat{f}$
		\item Exact conclusions about the prominence gap.
		\item How to specify prominence parameter $\tau$
		\item Algorithm details
	\end{enumerate}
\end{enumerate}
\subsection{Construct Complex}
\begin{enumerate}
	\item Need to learn about the various complexes, esp. their advantages / disadvantages for applications.
	\begin{enumerate}
		\item \v Cech
		\item VR
		\item Delaunay
		\item Alpha
		\item Witness
	\end{enumerate}
\end{enumerate}
\subsection{Persistent Homology}
\begin{enumerate}
	\item How to compute a filtration from point cloud?
	\begin{enumerate}
		\item Various ways to define simplices, Cech, VR, etc
		\item What if data not necessarily point cloud?
		\begin{enumerate}
			\item Image?
			\item 3D objects?
		\end{enumerate}
	\end{enumerate}
	\item Need to delve into the details of the PH algorithms
	\begin{enumerate}
		\item Like reading off the intervals.
		\item Different implementations.
	\end{enumerate}
	\item ALL the above pretty much FINISHED.
\end{enumerate}

\section{Clustering}
Point cloud (with coordinates)\\

Distance / dissimilarity matrix\\

Note: this seems to be a different idea of the distance function used in TDA, which is called lens in that context.\\

Barcode \textrightarrow merge tree \textrightarrow dendrogram

\subsection{Mode Seeking Paradigm}
Problems:
\begin{enumerate}
	\item Noisy estimator
	\item Neighborhood graph
\end{enumerate}
Solutions:
\begin{enumerate}
	\item Be proactive: smooth
	\item Be reactive: merge clusters after clustering\\
		  This leads to "topological persistence"
\end{enumerate}

Persistence for Model Seeking:\\
Probability density function $f$\\
\begin{itemize}
	\item Nested family (filtration) of of inverse images, or superlevel-sets $f^{-1}([t,+\infty))$ for 			  $t$ from $+\infty$ to $-\infty$
	\item Track evolution of "topology"
\end{itemize}
Similar stability theorem.\\
Seems to have relation with Morse theory. "If $f$ is Morse, then..."\\

\section{Topological Persistence}
Persistence diagram shows the "persistence" of the topological features.
Slight perturbation causes slight difference in persistence diagrams.

\section{Homology}
Definition: $h:X\mapsto Y$ is a homeomorphism if there exists a map $h^{-1}:Y\mapsto X$, s.t.
\begin{itemize}
	\item both are continuous
	\item
\end{itemize}

\section{How to deal with data}
\subsection{Networks}
Essentially 1-dimensional simplicial complex. Filter by weight.\\
Construct higher dimensional simplices. Ex: WRCF.\\
Mapping the nodes to points in finite metric space.
\subsection{Digital Images}
Natural cubical structure. Cubical complexes.\\
$c$ color variables, $N$ pixels/voxels, then $c\times N$-dimensional space, equipped a distance function to form a finite metric space.

\section{Construct (Simplicial) Complex}
\subsubsection{Various Ways}
\subsubsection{Reduction Techniques}
Heuristic ways to reduce the size of a filtered complex while keeping the PH unchanged.
\paragraph{Discrete Morse Theory}
Refer to \cite{Mischaikow2013} for theoretic details. NP complete, thus relies on heuristics to find partial matching to reduce the size.
\paragraph{Strong Collapses}
Refer to \cite{Barmak2012} for details.

\section{Computation of Persistence Homology}
This part is largely cited from \cite{Otter2017}.\\
How to keep track of how one feature "merges" to another?\\
Boundary matrix: the matrix representation of boundary operator.\\
We also need a total ordering compatible with the "filtration" in the following sense:
\begin{itemize}
	\item a face of a simplex precedes the simplex.
	\item a simplex in $K_i$ precedes simplices in $K_j$ for $j>i$, and not in $K_i$.
	\begin{itemize}
		\item this essentially means that we place the simplices by the order of "appearing".
	\end{itemize}
\end{itemize}

\subsection{Standard Algorithm}
\begin{itemize}
	\item Form the boundary matrix from the ordering.
	\item Reduction, which is essentially Gaussian elimination.
	\item Reading off intervals.
	\begin{enumerate}
		\item some details to do.
		\item degree: $\text{dg}(\sigma)=\text{smallest number }l\text{ s.t.}\sigma\in K_l$
		\item pair $(\sigma_i,\sigma_j)$ gives $[\text{dg}(\sigma_i),\text(\sigma_j))$
		\item unpaired extends to infinity.
	\end{enumerate}
	\item 
\end{itemize}

\subsection{Complexity}
In the worst case, which does exist, the algorithm has cubic complexity. Note that when sparse, not cubic.

\section{Statistical Interpretation}
\subsection{Problems}
\begin{enumerate}
	\item Compare outputs with null model.\\
	How to compare the different results?\\
	How to evaluate the significance of the data?
	\item Average over multiple realizations of a random model.
\end{enumerate}
\subsection{Statistical Analysis}
Statistical methods for PH addressed first time in \cite{Bubenik2007}. Three approaches.
\begin{enumerate}
	\item Topological properties of random simplicial complexes, viewed as null models.
	\item Properties of a metric space, whose points are persistence diagrams.
	\item "Features" of persistence diagrams.
\end{enumerate}
\subsubsection{Second Approach}
Key point: define an appropriate distance function between diagrams. The main object is the persistence diagram, which is in some sense isomorphic to a barcode.
\paragraph{Wasserstein Distance} is a popular way to defined distance. Details skipped.
\subsubsection{Third Approach}
Key point: map the space of persistence diagrams to analyzable spaces (e.g. Banach spaces). Persistence landscape\cite{Bubenik2015}, using space of algebraic functions\cite{Adcock2013}, kernalized techniques.

\section{Mapper}

\section{Why TDA?}
\subsection{Theoretically well understood}
\subsection{Qualitative data features}
\subsection{Computable via linear algebra}
\subsection{Robust under perturbation}

\section{Application: Possible Topics and Data Sets to Try}
\subsection{List of fields}
\begin{enumerate}
	\item Sensor network coverage.
	\item Proteins.
	\item 3-dimensional structure of DNA
	\item Robotics
	\item Signals in images
	\item Periodicity in time series
	\item Cancer
	\item Phylogenetics
	\item Natural images
	\item Self-similarity in geometry
	\item Materials science
	\item Financial networks
	\item Neuroscience
	\item Other networks
	\item Time-series output of dynamical systems
	\item Natural language analysis
\end{enumerate}
\subsection{Important Examples}
\subsubsection{Digital Image from Otter 17}
This example comes from \cite{Otter2017}.\\
Two approaches: Cubical Complexes and //TODO
\paragraph{Cubical Complexes}
Instead of simplices, one use cubical complexes to build the topological space.\\
Sounds like \v Cech?\\
\begin{enumerate}
	\item Every pixel is a point, join the adjacent points;
	\item Color the unit squares and cubes (only unit ones!);
	\item Label each pixel with grey scale, each edge with the max of point;
	\item Filter the topology by grey scale values;
	
\end{enumerate}
\subsection{Data Sets}
\subsection{Choice of Software}
List of some open source softwares available
\begin{enumerate}
	\item Javaplex\cite{Javaplex}
	\item Dionysus
	\item DIPHA
	\item Perseus
	\item GUDHI
	\item Ripser, the known best one, according to \cite{Otter2017}
\end{enumerate}


\newpage
\bibliography{Reference}
\bibliographystyle{ieeetr}

\end{document}