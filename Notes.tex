\documentclass[10pt,a4paper]{article}
\usepackage[utf8]{inputenc}
\usepackage{amsmath}
\usepackage{amsfonts}
\usepackage{amssymb}
\usepackage{textcomp}

\begin{document}

\section{List of works}
\subsection{Introduction and Summary}
\begin{enumerate}
	\item \cite{Otter2017} Roadmap for computation of PH, 2017\\
		  Thorough review of computation and some theory of PH.
\end{enumerate}
\subsection{Original work}
\subsubsection{TDA}
\subsubsection{Persistent Homology}
\subsection{Applications}

\section{Questions}
\subsection{Theory}
\subsubsection{Homology}
\begin{enumerate}
	\item What other "features" can be revealed by homology?\\
		  It seems that although Homology groups are informational\\
		  Betti numbers, which are the basis for barcode and other diagrams, do not tell us anything more that holes and voids.\\
		  
\end{enumerate}
\subsection{Computation}
\begin{enumerate}
	\item What is the bottleneck of the general computation process?
	\begin{enumerate}
		\item Is it the same for all the libraries?
		\item Possible optimization? Meaningful?
	\end{enumerate}	  
	\item What is the bottleneck for a specific data type?
	\item For a specific purpose?
\end{enumerate}
\subsection{Application}
\begin{enumerate}
	\item Search for a problem?\\
		  It seems that the methods is a not problem-solving-oriented.\\
		  But rather developed from math intuition.\\
		  Therefore seems bound to be less powerful than DL, which is clearly developed to solve problems.\\
		  
\end{enumerate}

\section{TODO List}
\subsection{Classic Dimensionality Reduction}
\begin{enumerate}
	\item Learn about the following:
	\begin{enumerate}
		\item PCA
		\item MDS
		\item isomap
	\end{enumerate}
	\item Maybe need to implement some of them
	\item Know the difference from a practical aspect
\end{enumerate}
\subsection{Clustering}
\begin{enumerate}
	\item Need to learn about the different approaches 
	\begin{enumerate}
		\item Variational
		\item Spectral
		\item Hierarchical
		\item Density thresholding
		\item Mode seeking
		\item Valley Seeking
	\end{enumerate}
	\item Need to know the specific applications
	\item Need to delve into one of them to see if possible for improvement
	\item Mathematical details about Mode Seeking
	\begin{enumerate}
		\item Hypotheses on the function of $f$ and estimator $\hat{f}$
		\item Exact conclusions about the prominence gap.
		\item How to specify prominence parameter $\tau$
		\item Algorithm details
	\end{enumerate}
\end{enumerate}
\subsubsection{Persistent Homology}
\begin{enumerate}
	\item How to compute a filtration from point cloud?
	\begin{enumerate}
		\item Various ways to define simplices, Cech, VR, etc
		\item What if data not necessarily point cloud?
		\begin{enumerate}
			\item Image?
			\item 3D objects?
		\end{enumerate}
	\end{enumerate}
	\item Need to delve into the details of the PH algorithms
	\begin{enumerate}
		\item Like reading off the intervals.
		\item Different implementations.
	\end{enumerate}
\end{enumerate}

\section{Clustering}
Point cloud (with coordinates)\\

Distance / dissimilarity matrix\\

Note: this seems to be a different idea of the distance function used in TDA, which is called lens in that context.\\

Barcode \textrightarrow merge tree \textrightarrow dendrogram

\subsection{Mode Seeking Paradigm}
Problems:
\begin{enumerate}
	\item Noisy estimator
	\item Neighborhood graph
\end{enumerate}
Solutions:
\begin{enumerate}
	\item Be proactive: smooth
	\item Be reactive: merge clusters after clustering\\
		  This leads to "topological persistence"
\end{enumerate}

Persistence for Model Seeking:\\
Probability density function $f$\\
\begin{itemize}
	\item Nested family (filtration) of of inverse images, or superlevel-sets $f^{-1}([t,+\infty))$ for 			  $t$ from $+\infty$ to $-\infty$
	\item Track evolution of "topology"
\end{itemize}
Similar stability theorem.\\
Seems to have relation with Morse theory. "If $f$ is Morse, then..."\\

\section{Topological Persistence}
Persistence diagram shows the "persistence" of the topological features.
Slight perturbation causes slight difference in persistence diagrams.

\section{Homology}
Definition: $h:X\mapsto Y$ is a homeomorphism if there exists a map $h^{-1}:Y\mapsto X$, s.t.
\begin{itemize}
	\item both are continuous
	\item
\end{itemize}



\section{Computation of Persistence Homology}
This part is largely cited from \cite{Otter2017}.\\
How to keep track of how one feature "merges" to another?\\
Boundary matrix: the matrix representation of boundary operator.\\
We also need a total ordering compatible with the "filtration" in the following sense:
\begin{itemize}
	\item a face of a simplex precedes the simplex.
	\item a simplex in $K_i$ precedes simplices in $K_j$ for $j>i$, and not in $K_i$.
	\begin{itemize}
		\item this essentially means that we place the simplices by the order of "appearing".
	\end{itemize}
\end{itemize}

Standard Algorithm:
\begin{itemize}
	\item Form the boundary matrix from the ordering.
	\item Reduction, which is essentially Gaussian elimination.
	\item Reading off intervals.
	\begin{enumerate}
		\item some details to do.
		\item degree: $\text{dg}(\sigma)=\text{smallest number }l\text{ s.t.}\sigma\in K_l$
		\item pair $(\sigma_i,\sigma_j)$ gives $[\text{dg}(\sigma_i),\text(\sigma_j))$
		\item unpaired extends to infinity.
	\end{enumerate}
	\item 
\end{itemize}

\section{Mapper}

\section{Why TDA?}
\subsection{Theoretically well understood}
\subsection{Qualitative data features}
\subsection{Computable via linear algebra}
\subsection{Robust under perturbation}

\section{Possible Topics and Data Sets to Try}
\begin{enumerate}
	\item Sensor network coverage.
	\item Proteins.
	\item 3-dimensional structure of DNA
	\item Robotics
	\item Signals in images
	\item Periodicity in time series
	\item Cancer
	\item Phylogenetics
	\item Natural images
	\item Self-similarity in geometry
	\item Materials science
	\item Financial networks
	\item Neuroscience
	\item Other networks
	\item Time-series output of dynamical systems
	\item Natural language analysis
\end{enumerate}

\newpage
\bibliography{Reference}
\bibliographystyle{ieeetr}

\end{document}