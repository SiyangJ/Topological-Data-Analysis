\documentclass[10pt,a4paper]{report}
\usepackage[utf8]{inputenc}
\usepackage{amsmath}
\usepackage{amsfonts}
\usepackage{amssymb}
\begin{document}

\section{Basics}
\paragraph{Topological space:} defined as a set of points, along with a set of neighborhoods.\\
Satisfy a set of axioms relating the points and the neighborhood.\\
Motivation: most general notion of a mathematical space that allows for \\
\begin{itemize}
	\item continuity, 
	\item connectedness,
	\item convergence
\end{itemize}
Extension: manifolds, metric spaces, etc\\
Commonly used: defined in terms of open sets.\\
More intuitive: neighborhoods.\\

$X$ a set, together with $\mathbf{N}:X\mapsto 2^{2^{X}}$, $N\in\mathbf{N}(x)$ is a neighborhood of $x$, satisfying
\begin{enumerate}
	\item Point in its own neighborhood;
	\item Superset of neighborhood is a neighborhood, $N\subseteq M\implies M\in\mathbf{N}(x)$;
	\item Intersection of neighborhoods is a neighborhood, $\forall N,M\in\mathbf{N}(x),N\cap M\in				  \mathbf{N}(x)$
	\item $\forall N\in\mathbf{N}(x)\exists M,s.t. y\in M\implies N\in\mathbf{N}(y)$
\end{enumerate}

Open sets: $x\in U\implies U\in\mathbf{N}(x)$\\
\\
Equivalent definition with respect to open sets:\\
$(X,\tau),\tau\subseteq 2^{2^{X}}$, satisfying:
\begin{enumerate}
	\item $\o,X\subseteq \tau$
	\item $N_{i}\in\tau,\bigcup_{i}N_{i}\in\tau$
	\item $N_{i}\in\tau,\bigcap_{i}N_{i}\in\tau,i$ finite
\end{enumerate}
These are called open sets.\\
\\
\paragraph{Continuous:}the inverse image of every open set is an open set.

\paragraph{Homotopy:}
Intuitively from the idea of continuous deformation; is strictly weaker than homeomorphism.

\section{Homology}
Intuitive view:\\
\paragraph{Path:} continuous map $[0,1]\mapsto X$; $x\sim y$ if $\gamma(0)=x,\gamma(1)=y$.\\
This is clearly an equivalence relation. Therefore, the path connected components of $X$ are equivalence classes under $\sim$.
\subsection{Simplicial Complexes}
General idea is that simplicial complexes extend the notion of graph to include higher dimensional components in it.\\
Turn simplicial complexes to topological spaces: "embed" it into Euclidean space.\\
Embedding: map points to coordinates and all points affinely independent.\\
Induced map: $\hat{f}:K\to 2^{\mathbb{R}^d},\{ v_0,v_1,...,v_r\}\mapsto Conv(f(v_0),...,f(v_r))$\\
All embeddings $f:K\to\mathbb{R}^d,g:K\to\mathbb{R}^{d'}$, $\hat{f}$ and $\hat{g}$ are homeomorphic.

\end{document}