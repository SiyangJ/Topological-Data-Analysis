\documentclass[10pt,a4paper]{article}
\usepackage[utf8]{inputenc}
\usepackage{amsmath}
\usepackage{amsfonts}
\usepackage{amssymb}
\begin{document}

\section{Basics}
\paragraph{Topological space:} defined as a set of points, along with a set of neighborhoods.\\
Satisfy a set of axioms relating the points and the neighborhood.\\
Motivation: most general notion of a mathematical space that allows for \\
\begin{itemize}
	\item continuity, 
	\item connectedness,
	\item convergence
\end{itemize}
Extension: manifolds, metric spaces, etc\\
Commonly used: defined in terms of open sets.\\
More intuitive: neighborhoods.\\

$X$ a set, together with $\mathbf{N}:X\mapsto 2^{2^{X}}$, $N\in\mathbf{N}(x)$ is a neighborhood of $x$, satisfying
\begin{enumerate}
	\item Point in its own neighborhood;
	\item Superset of neighborhood is a neighborhood, $N\subseteq M\implies M\in\mathbf{N}(x)$;
	\item Intersection of neighborhoods is a neighborhood, $\forall N,M\in\mathbf{N}(x),N\cap M\in				  \mathbf{N}(x)$
	\item $\forall N\in\mathbf{N}(x)\exists M,s.t. y\in M\implies N\in\mathbf{N}(y)$
\end{enumerate}

Open sets: $x\in U\implies U\in\mathbf{N}(x)$\\
\\
Equivalent definition with respect to open sets:\\
$(X,\tau),\tau\subseteq 2^{2^{X}}$, satisfying:
\begin{enumerate}
	\item $\o,X\subseteq \tau$
	\item $N_{i}\in\tau,\bigcup_{i}N_{i}\in\tau$
	\item $N_{i}\in\tau,\bigcap_{i}N_{i}\in\tau,i$ finite
\end{enumerate}
These are called open sets.\\
\\
\paragraph{Continuous:}the inverse image of every open set is an open set.

\paragraph{Homotopy:}
Intuitively from the idea of continuous deformation; is strictly weaker than homeomorphism.

\section{Homology}
Intuitive view:\\
\paragraph{Path:} continuous map $[0,1]\mapsto X$; $x\sim y$ if $\gamma(0)=x,\gamma(1)=y$.\\
This is clearly an equivalence relation. Therefore, the path connected components of $X$ are equivalence classes under $\sim$.
\subsection{Simplicial Complexes}
General idea is that simplicial complexes extend the notion of graph to include higher dimensional components in it.\\
Turn simplicial complexes to topological spaces: "embed" it into Euclidean space.\\
Embedding: map points to coordinates and all points affinely independent.\\
Induced map: $\hat{f}:K\to 2^{\mathbb{R}^d},\{ v_0,v_1,...,v_r\}\mapsto Conv(f(v_0),...,f(v_r))$\\
All embeddings $f:K\to\mathbb{R}^d,g:K\to\mathbb{R}^{d'}$, $\hat{f}$ and $\hat{g}$ are homeomorphic.\\
Underlying space, $|K|$, the image of $K$ through embedding, unique up to homeomorphism.\\
Triangulability: $X$ is triangulable if $\exists K,h:X\to|K|$, $h$ is homeomorphism.\\
Simplicial map: combinatorial equivalence of continuous map.\\
Topological realization of a map: simplicial $f:K\to L$ induces continuous $|f|:|K|\to|L|$\\
Reverse is not necessarily true.\\
Simplicial Approximation: continuous $f:|K|\to|L|$ is homotopic to $|f'|:|K|\to|L|$, where $f':K'\to L$, for some simplicial subdivision $K'$ of $K$.\\
Intuition: by dividing the simplicial complex sufficiently many times, we can approximate the topological space.
\subsection{Simplicial Homology}
Orientation: Ordering of vertices, unique up to even permutation, negative by odd permutation.
\subsubsection{Chains}
$K$ finite simplicial complex and $k$ a fixed field. Given $r\in\mathbb{N}$, we are interested in the $k-$linear combinations (formal sums) of $r-$simplices in $K$.\\
The linear space of all the formal sums for each $r$ is a chain space of $k-$chains.\\
Note: Seems that the field $k$ can be relaxed to a ring. Can study later.\\
\subsubsection{Boundary Operator}
Remove one vertex from a simplex, we get a face of the simplex. The linear combination together with interchanging orientation is the boundary.\\
$\partial_r:C_r(K,k)\to C_{r-1}(K,k)$\\
$[v_0,...,v_r]\mapsto\sum_{j=0}^{r}(-1)^{j}[v_0,...\hat{v_j}...,v_r]$\\
Clearly $\partial_r$ is distributive.\\
Note:$\partial^2=0$, and therefore is nilpotent. (Actually it's $\partial_r\circ\partial_{r+1}=0$.
\subsubsection{Homology Groups}
Important motivation: want to find "cycles modulo the boundaries".\\
Note: need to further think about this.\\
r-cycles: $Z_r(K,k):=\text{ker}\partial$\\
r-boundaries: $Z_r(K,k):=\text{im}\partial_{r+1}$\\
Homology group: $Z_r(K,k)/Z_r(K,k)$\\
\subsubsection{Algorithm for Homology}
Since it's a vector space, it's isomorphic to $k^{\beta_r}$,\\
Note: when relaxed to ring, it's a module, and therefore we still have the fundamental theorem for modules to decompose to a torsion part and a torsion-free part. Still have some kind of $\beta_r$.\\
Note: everything is linear space / linear transformation.\\
Matrix form $M_r$ of $\partial_r$ for each $r$:\\
$\#K_r$ columns and $\#K_{r-1}$ rows, $\beta_r=\#K_r-\text{rank}M_r-\text{rank}M_{r+1}$, just compute ranks to get the dimension, which I have learned and don't want to go into details.\\
\subsubsection{Morphisms}
Operator on spaces: $H_r:K\mapsto H_r(K,k)$, which we want to extend to maps as well.\\
Idea of a functor? Some category stuff?\\
Chain level: simplicial map induces a chain map. Details omitted.\\
The chain map commutes with boundary operator.\\
Functoriality:\\
\subsection{From simplicial complexes to topological spaces}
Theorem: $X$ triangulable, then, $\forall K,L,H_r(K,k)\simeq H_r(L,k)$.\\
Conclusion: homology groups of triangulable spaces, and the morphisms between them, are uniquely defined, for different ways of triangulation. Moreover, morphisms are invariant under homotopy.\\
Corollary: $X\sim Y\implies H_r(X,k)\simeq H_r(Y,k)$.\\
However, homology does not completely characterize the topology of a space. Thus still much weaker than homeomorphism.
\section{Exercise}
Compute the homology groups, and in particular, the Betti numbers.\\
Often we only want the Betti numbers, so choose the simplest field, $\mathbb{Z}_2$, whereby we can ignore the orientation.\\
The key is
\begin{enumerate}
	\item triangulation
	\item identify the structures
	\item computationally, matrix and linear algebra
\end{enumerate}
Questions:
\begin{enumerate}
	\item Circle, $\mathbb{S}^1:\beta_0=1,\beta_1=1$
	\item Disk, $\mathbb{B}^2:\beta_0=1,\beta_1=1,\beta_2=0,etc$, same homology groups as a single point. In fact, homotopy equivalent to a point.
	\item Cylinder, $\mathbb{S}^1\times [0,1]:\beta_0=1,\beta_1=1,\beta_2=0$, homotopy equivalent to a circle.
	\item Sphere, $\mathbb{S}^2:\beta_0=1,\beta_1=0,\beta_2=1$
	\item Ball, $\mathbb{B}^3:\beta_0=1,\beta_1=0,\beta_2=0,etc$, homotopy equivalent to a single point.
	\item Torus, see online for triangulation, $\mathbb{T}:\beta_0=1,\beta_1=2,\beta_2=1,\beta_3=0$
	\item Homology for sphere $\mathbb{S}^d$, $\beta_0=1,\beta_i=0,\beta_d=1$
\end{enumerate}
Brouwer's fixed point theorem:\\
$\forall$continuous $f:\mathbb{B}^2\to\mathbb{B}^2,\exists p\in\mathbb{B}^2,f(p)=p$.\\
Hairy ball theorem:\\
For $d$ even, $\forall$continuous tangent vector field $V:\mathbb{S}^d\to\mathbb{R}^d,\exists p\in\mathbb{S}^d,$ s.t. $V(p)=0$.\\

\end{document}